% !TeX root = ntfs.tex
\section{Introduction}
Computers have always been used as a tool for storing and accessing information. One could argue they act as a form of digital library. A centralized place which contains a plenitude of data.
Though just having data does not do much good. It needs to quickly accessible when needed. As of such there  needs to be a form of organization. What in the context of a library might be an indexing system, is a file system to a computer.\\
A file system ensures quick and organized access to data without needing to be familiar with the underlying storage medium (e.g., solid state disk, hard drive, or tape drive).
Furthermore a file system should be resilient to system crashes, which essentially means preventing corruption of the file system, as this would inevitably lead to data loss.\\
Being an integral part of computing a file system is a core component of all modern operating systems, including Microsoft Windows. Since 1993\cite{Custer:1994:IWN} Microsoft has shipped all desktop and server Windows operating systems with a file system called \textit{New Technology File System} (\textit{NTFS}). In the following paper we will take a closer look at how NTFS achieves the aforementioned requirements for a file system: First it will be described how NTFS lays out the available storage in section \ref{sec:Cluster}, after which the data structure for organizing stored data will be looked at in section \ref{sec:MFT}. Lastly the resilience aspect of NTFS will be highlighted in section \ref{sec:Recovery}.